% arara: xelatex: { shell : yes }
% arara: biber
% arara: xelatex: { shell : yes }
% arara: xelatex: { shell : yes }
% překládejte pomocí příkazu xelatex
\documentclass{article}

\usepackage{polyglossia}
\setmainlanguage{czech}
\usepackage{csquotes}
\usepackage{graphicx}
\usepackage{url}
\usepackage[hidelinks]{hyperref}
\usepackage[style=iso-numeric,backend=biber]{biblatex}
\addbibresource{ref.bib}

%-------- odkomentujte pokud chcete pro seznam literatury používat biblatex, číselné odkazy  -------
%\usepackage[style=iso-numeric]{biblatex}
%\addbibresource{mybibliography.bib}
%----------------------------------------------------------------------------------------------------

\title{Automatický provisioning herních serverů pomocí cloudové image} %doplňte název bakalářské práce
\author{\small Autor: Matyáš Ješina\\\small Vedoucí: Ing. Tomáš Vondra, Ph.D.\\ \small Studijní obor: Webové a softwarové inženýrství} %doplňte své jméno, jméno vedoucího a svůj studijní obor
\date{\small \url{jesinmat@fit.cvut.cz}} %doplňte svůj email

\begin{document}

\maketitle              

\paragraph{Klíčová slova}{herní server, počítačová hra, cloud, automatizace, nasazování systému}
%doplňte cca 5 až 10 klíčových slov oddělených čárkou
%klíčová slova jsou odborné termíny popisující zaměření práce
%mohou být vyjádřeny i více slovy (např. "konečný automat")

\section{Úvod}
%popis problému, který práce řeší + lze popsat i co práce neřeší (pokud je to dle vás potřeba)
%důvody, proč je zajímavé/důležité se problémem zabývat, přínosy práce
%vaše motivace pro výběr tématu (je-li dle vás zajímavá)
%návaznost na jiné závěrečné práce 
%popis struktury práce (stručný popis následujících sekcí)3
Počítačové hry se těší velké popularitě. S rozšířením internetu začaly vznikat také hry
pro více hráčů, které se rychle dostaly do čela žebříčků oblíbenosti a dnes jsou zábavou pro stovky milionů
hráčů po celém světě.

Mnohé z těchto her umožňují uživatelům vytvořit vlastní herní servery, na kterých je možné hrát s přáteli
či jinou komunitou.

Pokud chce uživatel zprovoznit herní server, měl by takový postup být jednoduchý a rychlý.
Existuje zde například možnost provozu serveru na vlastním počítači, zde je však kvalita herního zážitku 
ovlivněna konfigurací systému a internetovým připojením. Také je často potřeba pokročilého nastavení
směrovače, který herní server z domácí sítě zpřístupní do internetu.

Tato práce se zaměřuje na další možnost provozu těchto serverů, a to v cloudovém prostředí. Uživatel se tak nemusí zabývat 
kvalitou internetového připojení či manuálním nastavováním síťových prvků. Herní servery
pro menší počet hráčů jsou často vytvářeny a rušeny, nasazení v cloudu tedy představuje ideální způsob provozu,
kde jsou tyto operace jednoduché a automatizovatelné.

Cílem práce je vytvořit obraz systému, který bude pro toto použití vhodný. Uživatel bude mít možnost vybrat
požadovaný herní server a systém provede všechny operace potřebné k jeho zprovoznění.

Výsledek práce bude prospěšný zejména pro stávající uživatele cloudových služeb, kteří mají alespoň minimální
zkušenosti s nasazováním obrazů systémů. S minimální interakcí budou mít možnost spustit herní server dle svého
výběru bez složité instalace a konfigurace. Pokročilí uživatelé využijí možnosti automatizace celého procesu,
která jim zaručí rychlé spuštění vybraného serveru za pomoci několika příkazů.

Vytvořený obraz musí být snadno dostupný a zdokumentovaný, jednoduchý na nasazení, s minimální náročností 
na systémové prostředky. Bude jej také možné využít v komerčním prostředí.

\section{Cíl(e) práce}
%stanovte si cíl(e) pro svou práci (naplnění pak zhodnotíte v závěru)
%lze rozdělit na hlavní cíle a dílčí (pokud máte více cílů)
Hlavním cílem této práce je vytvořit obraz systému, který bude schopný provozovat herní servery v cloudovém prostředí.
Tento systém musí být jednoduchý na zprovoznění i úpravy. Bude tedy ideálním kandidátem pro uživatele se základní znalostí
nasazování serverů v cloudu, který nechce provozovat herní server na vlastní infrastruktuře.
Práce se zaměří na herní servery pro menší množství hráčů (přibližně do 50), které obecně vyžadují méně systémových prostředků a nastavení
a proces jejich vytváření je tak jednoduše automatizovatelný.

V první části práce je tedy nutné analyzovat dostupné možnosti a najít výhody a nevýhody daných řešení. Dále je potřeba vybrat
vhodný operační systém, který splňuje dané požadavky. V dalším kroku budou porovnány různé možnosti provozu herních serverů
na daném systému a jejich automatizace.

V praktické části bude implementována součást pro automatickou instalaci herních serverů a jejich provoz. Jelikož je kladen důraz
na jednoduchost provozu, bude tento program pracovat s minimální interakcí uživatele, případně plně automaticky.
Tato součást také musí být schopna spouštět a zastavovat herní server, bude-li to nutné.

Jelikož se bude po nasazení jednat o veřejně dostupný systém, je zde důležitým prvkem jeho bezpečnost. Budou tedy vyhodnoceny možnosti
jeho zabezpečení, zahrnující například vzdálený přístup.

\section{Současný stav řešení problému}
%teoretická část, popis definic a zavedení používaných pojmů/metod atd. 
%popis dosavadních poznatků (rešerše) k danému problému (kdo a jak problém řešil, k čemu došel, výhody a nevýhody) 
%tato kapitola prokazuje vaši znalost daného tématu, zde budete nejvíce citovat
Vzhledem k množství herních serverů existují různé způsoby jejich provozu. Aplikace nabízené přes platformu Steam lze spravovat pomocí nástroje
SteamCMD \cite{steamcmd}, avšak tento způsob neumožňuje provozovat herní servery jiných společností.
Z tohoto důvodu existují aplikace na správu herních serverů, které umožňují spouštět servery od mnoha společností uživatelsky přívětivou cestou.
Mezi existující nástroje se řadí například:
\begin{itemize}
    \item GameCP -- placený nástroj pro správu herních serverů,
    \item Pterodactyl -- aplikace pro správu mnoha herních serverů na různých systémech,
    \item PufferPanel -- podobná aplikace jako Pterodactyl, s menším množstvím podporovaných her,
    \item LinuxGSM -- sada skriptů pro správu herních serverů,
    \item Open Game Panel -- správce herních serverů, nyní již neudržovaný.
\end{itemize}
Mnohé z těchto aplikací jsou vhodné pro správu velkého množství herních serverů, které běží na různých systémech. Díky webovému rozhraní umožňují snadný
přístup pro administrátory, případně i zavedení platebního systému pro zákazníky. Jsou tedy vhodné převážně pro komerční subjekty, které se zabývají
provozem herních serverů pro zákazníky. Většina jejich funkcí je pro tuto práci zbytečná.


\section{Vaše řešení daného problému}
%zde by měl být popsán váš přístup k řešení daného problému, odůvodnění zvolených technik, experimenty, výpočty, testování, popis implementace apod.
%jedná se o tvůrčí část práce, tj. musí být poznat, co je vaše práce
Požadované řešení lze rozdělit na dvě části:
\begin{itemize}
    \item Prostředí, ve kterém aplikace běží -- toto může představovat celý operační systém, případně pouze kontejner.
    \item Správce herního serveru -- aplikace, která se stará o instalaci a provoz daného herního serveru.
\end{itemize}
V prvním kroku je nutné zvolit vhodný operační systém. Mělo by se jednat o rozšířený a snadno použitelný systém,
který lze provozovat bez vysokých nároků na systémové prostředky. Tento systém by měl být volně použitelný i pro komerční účely.
Je žádoucí maximální bezpečnost a možnost provozu bez ohledu na hostitelský operační systém, proto byl zvolen kompletní obraz operačního systému
místo kontejneru.

Pro provoz serverů byla zvolena distribuce TurnKey GNU/Linux \cite{turnkey}, založená na Debianu. TurnKey se specializuje na provoz systému ve virtuálním prostředí,
umožňuje vytvářet obrazy systému pro populární cloudové platformy (Amazon Web Services, OpenStack) pomocí připravených skriptů a obsahuje
nástroje pro automatické nastavení systému bez zásahu uživatele.

Vývojáři TurnKey nabízí možnost zveřejnění vzniklého obrazu systému v repozitáři TurnKey Hub, odkud jsou obrazy zdarma nabízeny zájemcům
včetně možnosti jejich okamžitého nasazení na cloudové služby. Pro zveřejnění v tomto repozitáři musí obraz obsahovat pouze svobodný a otevřený
software, aby nedošlo k porušení licenčních podmínek.

Dále je třeba určit aplikaci pro správu herních serverů. Také tato aplikace musí mít možnost plné automatizace a provozu serverů bez zásahu uživatele.
Musí běžet na zvoleném operačním systému a splňovat podmínky pro zveřejnění v TurnKey Hub. Pro správu byla zvolena aplikace LinuxGSM \cite{linuxgsm}, která umožňuje
automatizaci celého procesu a splňuje všechny požadavky. LinuxGSM podporuje přes 100 různých herních serverů a díky otevřenosti zdrojového kódu je možné přidat podporu
i pro budoucí hry.

Pro integraci těchto částí byly vytvořeny dva projekty. Prvním z nich je sada skriptů pro automatizaci použití LinuxGSM. Umožňuje jedním příkazem provést stažení LinuxGSM,
jeho nainstalování, zvolení hry, i její následnou základní konfiguraci. O samotnou instalaci a spuštění hry se pak stará zmíněný správce herních serverů.
Tato sada není závislá na konkrétním systému a lze ji použít i v jiných případech.

Druhou součástí je pak samotný obraz operačního systému. Obsahuje interaktivního grafického průvodce pro uživatele, kterému umožní vybrat požadovaný herní server a provést
jeho základní nastavení, po kterém dojde ke spuštění automatické instalace. V případě provozu v cloudovém prostředí pak umožňuje převzít potřebné informace z externího zdroje
a provést instalaci herního serveru zcela bez zásahu uživatele. Tento obraz lze sestavit pomocí připravených skriptů v systému TurnKey GNU/Linux.


\section{Závěr}
%shrnutí cíle (cílů) práce a zhodnocení jeho (jejich) naplnění
%uvedení dosažených výsledků, komentář k využití daného řešení v praxi
%možné podněty pro navazující práce (výhled do budoucna)
Výsledkem této práce je obraz systému, který je vhodný pro rychlé nasazení herního serveru v cloudu.
Vhodná volba základní distribuce mi umožnila vytvořit jednoduchý a snadno rozšiřitelný systém,
který je vhodný zejména pro uživatele bez pokročilých technických znalostí.

Díky použité aplikaci pro správu herních serverů je možné používat systém pro velké množství populárních her.
Vytvořené skripty se starají o jejich automatickou instalaci a počáteční nastavení, celý systém je tak
možné provozovat v cloudu s minimální uživatelskou interakcí.

Systém automaticky po spuštění přenastaví administrátorské heslo a přístupové klíče, aby zamezil vstupu
nepovolaných osob. Pro stahování herních serverů využívá výhradně zabezpečených protokolů.
Tím je zajištěna bezpečnost proti základním typům útoků.

Obraz je zveřejněný na serveru TurnKey Hub, kde je také k dispozici návod k jeho použití v angličtině.
Je možné jednoduše přidat podporu pro další herní servery bez nutnosti zásahu do systému pomocí externích skriptů.

V případě dalšího vývoje je možné systém rozšířit o pokročilé sledování stavu herního serveru. Použitý LinuxGSM podporuje
zobrazení základních vlastností serveru, další nástroje (např. GameDig) pak umožňují získat informace o serveru i z externího
prostředí. Pro komerční použití by bylo vhodné doplnit systém o možnost podrobné uživatelské konfigurace bez nutnosti FTP přístupu.

% ---- Seznam literatury ----
% použít můžete prostředí "thebibliography" nebo "biblatex" -- (ne)výhody jednotlivých řešení budou probrány na cvičení

\printbibliography

\end{document}
