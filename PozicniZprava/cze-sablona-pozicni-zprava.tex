% arara: xelatex: { shell : yes }
% arara: biber
% arara: xelatex: { shell : yes }
% arara: xelatex: { shell : yes }
% překládejte pomocí příkazu xelatex
\documentclass{article}

\usepackage{polyglossia}
\setmainlanguage{czech}
\usepackage{csquotes}
\usepackage{graphicx}
\usepackage{url}
\usepackage[style=iso-numeric,backend=biber]{biblatex}
\addbibresource{mybibliographyfile.bib}
\addbibresource{ref.bib}

%-------- odkomentujte pokud chcete pro seznam literatury používat biblatex, číselné odkazy  -------
%\usepackage[style=iso-numeric]{biblatex}
%\addbibresource{mybibliography.bib}
%----------------------------------------------------------------------------------------------------

\title{Automatický provisioning herních serverů pomocí cloudové image} %doplňte název bakalářské práce
\author{\small Autor: Matyáš Ješina\\\small Vedoucí: Ing. Tomáš Vondra, Ph.D.\\ \small Studijní obor: Webové a softwarové inženýrství} %doplňte své jméno, jméno vedoucího a svůj studijní obor
\date{\small \url{jesinmat@fit.cvut.cz}} %doplňte svůj email

\begin{document}

\maketitle              

\paragraph{Klíčová slova}{herní server, počítačová hra, cloud, automatizace, nasazování systému}
%doplňte cca 5 až 10 klíčových slov oddělených čárkou
%klíčová slova jsou odborné termíny popisující zaměření práce
%mohou být vyjádřeny i více slovy (např. "konečný automat")

\section{Úvod}
%popis problému, který práce řeší + lze popsat i co práce neřeší (pokud je to dle vás potřeba)
%důvody, proč je zajímavé/důležité se problémem zabývat, přínosy práce
%vaše motivace pro výběr tématu (je-li dle vás zajímavá)
%návaznost na jiné závěrečné práce 
%popis struktury práce (stručný popis následujících sekcí)

\section{Cíl(e) práce}
%stanovte si cíl(e) pro svou práci (naplnění pak zhodnotíte v závěru)
%lze rozdělit na hlavní cíle a dílčí (pokud máte více cílů)

\section{Současný stav řešení problému}
%teoretická část, popis definic a zavedení používaných pojmů/metod atd. 
%popis dosavadních poznatků (rešerše) k danému problému (kdo a jak problém řešil, k čemu došel, výhody a nevýhody) 
%tato kapitola prokazuje vaši znalost daného tématu, zde budete nejvíce citovat

\section{Vaše řešení daného problému}
%zde by měl být popsán váš přístup k řešení daného problému, odůvodnění zvolených technik, experimenty, výpočty, testování, popis implementace apod.
%jedná se o tvůrčí část práce, tj. musí být poznat, co je vaše práce

\section{Závěr}
%shrnutí cíle (cílů) práce a zhodnocení jeho (jejich) naplnění
%uvedení dosažených výsledků, komentář k využití daného řešení v praxi
%možné podněty pro navazující práce (výhled do budoucna)

% ---- Seznam literatury ----
% použít můžete prostředí "thebibliography" nebo "biblatex" -- (ne)výhody jednotlivých řešení budou probrány na cvičení

\printbibliography

% BIBLATEX -- odkomentujte, pokud chcete pro seznam literatury používat biblatex
%\printbibliography

\end{document}
