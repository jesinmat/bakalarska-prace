\chapter[Analýza a návrh]{Analýza a návrh}

\section{Popis problému}

Vzhledem k rostoucí popularitě cloudových služeb \cite{statista_cloud_revenue} existuje množství zdrojů, ze kterých lze čerpat.
Herní servery se oproti ostatním typům běžných cloudových aplikací odlišují svojí vysokou náročností
na systémové prostředky, mimo jiné například požadavkem na nízkou latenci.

Při provozování cloudového serveru pro velké množství hráčů je nutné zajistit správné fungování infrastruktury,
jako je například správné vyvažování zátěže mezi jednotlivé servery či automatický výběr nejvhodnějšího serveru
pro klienta. Dle \cite{building_cloud_mog_server} je již od několika desítek současně připojených klientů vhodné
používat systém pro rozložení zátěže. Tyto problémy zde není nutné řešit -- výsledek práce má sloužit jako jednoduchý a rychlý
způsob nasazení herního serveru, není tedy z principu vhodný pro dlouhodobé obsluhování velkého množství hráčů.

U poskytovatelů cloudových serverů je možné zvolit množství systémových prostředků, které bude mít aplikace k dispozici.
Pokud sledujeme vytížení herních serverů v čase, můžeme spatřit jisté vzory, například nárůst hráčů ve večerních hodinách.
Jedná-li se o velký rozdíl v množství uživatelů, je nutné dynamicky navyšovat systémové prostředky \cite{efficient_resources}.
Stejně jako dříve zmíněný problém se i tento týká převážně aplikací pro velké množství uživatelů, v rámci této práce tedy tento problém není uvažován.
Již přidělené prostředky nemůže vytvořený systém nijak ovlivnit, jejich výběr bude tedy ponechán na uživateli před spuštěním.

Důležitým prvkem kteréhokoliv systému je zabezpečení. Aplikace musí být s důrazem na bezpečnost nejen provozována,
ale i vytvářena \cite{securing_cloud}. Bezpečnostní nedostatek může pro potenciálního útočníka znamenat možnost neoprávněného vstupu do systému.
Budou tedy prozkoumána dostupná bezpečností řešení pro cloudové aplikace.

Systém musí být schopný nainstalovat a spustit herní server automaticky, případně pouze s nezbytně nutnou interakcí uživatele.
Bude proveden průzkum dostupných možností pro automatickou instalaci herních serverů za účelem výběru vhodného řešení.
Následné spouštění i zastavování herních serverů musí být plně automatizované.

\section{Funkční požadavky}

Funkční požadavky definují požadované vlastnosti systému a určují akce, které je uživatel schopný v rámci aplikace vykonat.
Popisují jednotlivé vlastnosti, které musí systém splňovat, aby byl použitelný k provozu herního serveru.

\begin{itemize}
    \item \textbf{Spuštění systému} -- systém lze spustit automatizovatelnou cestou.
    \item \textbf{Prvotní nastavení} -- systém je schopen provést prvotní nastavení automaticky, či pomocí přehledné interakce s uživatelem. 
    \item \textbf{Výběr herního serveru} -- systém umožní uživateli vybrat požadovaný herní server a provést jeho základní nastavení.
    \item \textbf{Automatická instalace a spuštění herního serveru} -- systém bez další interakce s uživatelem nainstaluje a spustí herní server.
    \item \textbf{Nastavení herního serveru} -- systém umožní uživateli nastavit parametry zvoleného herního serveru.
\end{itemize}

\section{Nefunkční požadavky}

Cílem nefunkčních požadavků je zajistit kvalitu a stabilitu aplikace při jejím provozu. Mezi takové požadavky patří například bezpečnost, dostupnost, či rozšiřitelnost.
V případě provozu v cloudovém prostředí jsou mnohé tyto požadavky ovlivněny poskytovatelem služeb a není možné je ovlivnit v rámci instance systému.

\begin{itemize}
    \item \textbf{Podporované platformy} -- systém podporuje provoz na cloudových platformách a ve virtuálních strojích.
    \item \textbf{Bezpečnost} -- systém vynucuje dodržování bezpečnostních zásad a využívá šifrovanou komunikaci.
    \item \textbf{Dostupnost} -- herní server musí být dostupný v co nejkratším čase, a to i po výpadku.
    \item \textbf{Rozšiřitelnost} -- do aplikace lze jednoduše přidávat podporu pro nové herní servery.
\end{itemize}

Vzhledem k povaze problému nebude řešena škálovatelnost, která má vliv na plynulost provozu u systémů s větším počtem klientů.

\section{Řešení požadavků}

\subsection{Spuštění a prvotní nastavení systému}

Poskytovatelé cloudových služeb nabízejí webové rozhraní (API), které je určené k tvorbě instancí cloudových systémů a jejich konfiguraci.
Pomocí tohoto rozhraní lze spustit zvolený obraz systému automatizovatelnou cestou. Při vytváření nové instance lze v systému spustit uživatelský skript,
který umožní předat data do systému bez pozdější interakce s uživatelem.

Pro zajištění automatické konfigurace systému při prvním spuštění je nutné provést operace, které zahrnují například nastavení hesla k administrátorskému účtu,
provedení bezpečnostních aktualizací a nastavení předinstalovaných aplikací.
TurnKey GNU/Linux obsahuje mechanismus \mintinline{shell}{inithooks} \cite{inithooks}, který zajišťuje běh sady inicializačních skriptů při prvním spuštění systému.
V případě předcházejícího využití uživatelského skriptu pro předání potřebných dat tyto skripty automaticky nastaví systém do provozuschopného bez interakce s uživatelem.

Pokud uživatel spustí instanci systému v interaktivním prostředí bez předání potřebných dat, zobrazí \mintinline{shell}{inithooks} grafického průvodce prvním nastavením.
Uživatel tak může systém nakonfigurovat dle vlastního uvážení. Tento způsob inicializace je využit například v lokálních virtuálních strojích, u kterých je k dispozici
grafický výstup. Ukázka nastavení hesla pomocí tohoto mechanismu je na obrázku \ref{fig:inithooks-password}.

\begin{figure}[h]
    \centering
    \includegraphics[width=1\linewidth]{chapters/images/root-pass.pdf}
    \caption{Interaktivní nastavení systému při prvním spuštění \cite{turnkey}}
    \label{fig:inithooks-password}
\end{figure}

V případě spuštění bez grafického výstupu a uživatelských dat čeká systém na první přihlášení uživatele pomocí SSH s využitím přednastaveného klíče.
Některé cloudové služby také umožňují vygenerovat náhodné heslo pro přihlášení do systému. Po prvním přihlášení dojde k automatickému spuštění zmíněného grafického průvodce.

Systém dále obsahuje pokročilé interaktivní konfigurační prostředí, které se zobrazí automaticky po dokončení úvodního nastavení.
Prostředí umožňuje nastavit parametry systému a některých aplikací, bude vhodné prozkoumat možnosti integrace herní komponenty do této aplikace.
Ukázka tohoto prostředí je na obrázku \ref{fig:confconsole}.

\begin{figure}[h]
    \centering
    \includegraphics[width=1\linewidth]{chapters/images/confconsole.pdf}
    \caption{Konfigurační aplikace \cite{turnkey}}
    \label{fig:confconsole}
\end{figure}

\subsection{Výběr herního serveru}

K interaktivnímu výběru herního serveru dojde v případě, kdy uživatel nevybere požadovaný herní server pomocí uživatelského skriptu.
Do systému je tedy nutné přidat komponentu, která při prvním spuštění zobrazí seznam podporovaných herních serverů a umožní uživateli výběr.
K tomuto účelu bude využit mechanismus \mintinline{shell}{inithooks}, který zajistí zobrazení komponenty při prvním spuštění systému.

Seznam podporovaných herních serverů závisí na zvoleném správci herních serverů. Je tedy nezbytné, aby komponenta uměla získat seznam těchto serverů
a zohlednila jej ve výběru. Tento seznam není možné vložit do systému v době sestavení, neboť by byl výběr omezen v případě
přidání podpory pro nový herní server.

\subsection{Automatická instalace a spuštění herního serveru}

Po výběru herního serveru je nutné zvolený server nainstalovat. Pro tento účel bude využit správce herních serverů, který dokáže stáhnout potřebné soubory
a provést instalaci. Je tedy třeba vytvořit komponentu, která provede přípravu na instalaci serveru a následně spustí správce.

\subsection{Nastavení herního serveru}

Mnohé herní servery vyžadují po instalaci dodatečnou konfiguraci. Může být nutné zvolit herní prostředí, jméno administrátora, jméno serveru a mnohé další parametry.
Je nezbytné implementovat mechanismus, který umožní uživateli provést toto nastavení interaktivně, případně automaticky.
V případě spuštění bez možnosti interakce s uživatelem budou doplněny výchozí hodnoty, které uživatel může změnit při pozdějším přihlášení.

\subsection{Podporované platformy}

Systém musí být možné provozovat na různých cloudových platformách. Je třeba zavést podporu těchto platforem bez nutnosti manuálního přizpůsobování obrazu.
Dále je nezbytné podporovat obvyklou instalaci pro využití na standardních a virtuálních strojích.

\subsection{Bezpečnost}

Pro zachování bezpečnosti provozu musí být zajištěn bezpečný přístup do systému a k běžícím aplikacím. Dle \cite{securing_cloud} budou
pro komunikaci využity šifrované kanály, autentizace musí zajistit, aby se v žádném bodě provozu systému nemohl do systému přihlásit uživatel bez znalosti potřebných údajů.
Budou identifikována další možná bezpečnostní opatření a provedeny potřebné kroky k zajištění bezpečnosti provozu systému.

\subsection{Dostupnost}

Přestože samotná dostupnost systému je značně ovlivněna poskytovatelem cloudových služeb, lze provést některé optimalizační kroky v rámci systému.
Pokud nastane výpadek služby či dojde k restartování systému, musí automaticky dojít k opětovnému spuštění herního serveru, aby byla obnovena dostupnost i v případě
momentální absence administrátora serveru.

\subsection{Rozšiřitelnost}

Pro zajištění podpory herních serverů je nutné vždy instalovat jejich nejnovější verzi, kterou používají i klientská zařízení.
Vzhledem k četnosti aktualizací některých serverů je třeba zajistit, aby nebylo pro jejich podporu nutné znovu sestavovat obraz systému.
Bez opětovného sestavení musí být také možné přidat podporu pro nové herní servery. 

\section{Přístup k systému}

K zajištění bezpečnosti musí být omezen přístup k systému. Poskytovatelé cloudových služeb
umožňují omezit množinu portů, které budou na systému dostupné. Pro správu systému a připojení k hernímu
serveru je však nutné umožňit uživatelům přístup k daným síťovým rozhraním. Tato část popisuje způsob jejich zabezpečení.

TurnKey GNU/Linux standardně obsahuje webové rozhraní,
díky kterému je možné spravovat systém bez nutnosti přihlašování pomocí SSH. Pro přístup k webové administraci
je využito šifrované zabezpečení pomocí protokolu HTTPS, potřebný certifikát je vygenerován při prvním spuštění systému.
Tento certifikát je možné nahradit bezplatným certifikátem Let's Encrypt \cite{lets_encrypt} pomocí konfigurační aplikace.
Do webového prostředí se uživatel přihlašuje pomocí hesla zvoleného při prvním spuštění systému. 

TurnKey GNU/Linux
neobsahuje výchozí hesla pro služby, není tedy možné přihlásit se do neinicializovaných systémů bez oprávnění.
Uživatel při přihlášení do systému prokazuje svoji totožnost pomocí náhodně vygenerovaného hesla nebo klíče, které
jsou zvoleny před spuštěním systému. 

Pokud herní server podporuje vzdálený přístup do administrace, musí být při jeho instalaci uživatel vyzván ke změně výchozího hesla.
V případě jeho napadení útočníkem je nutné omezit dopad na systém. Bude tedy vhodné provozovat herní server s omezenými oprávněními. 
