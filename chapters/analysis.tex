\chapter[Analýza a návrh]{Analýza a návrh}

\section{Požadavky}

Vzhledem k rostoucí popularitě cloudových služeb \cite{statista_cloud_revenue} existuje množství zdrojů, ze kterých lze čerpat.
Herní servery se oproti ostatním typům běžných cloudových aplikací odlišují svojí vysokou náročností
na systémové prostředky, mimo jiné například požadavkem na nízkou latenci.

Při provozování cloudového serveru pro velké množství hráčů je nutné zajistit správné fungování infrastruktury,
jako je například správné vyvažování zátěže mezi jednotlivé servery či automatický výběr nejvhodnějšího serveru
pro klienta \cite{building_cloud_mog_server}. Tyto problémy zde není nutné řešit -- výsledek práce má sloužit jako jednoduchý a rychlý
způsob nasazení herního serveru, není tedy z principu vhodný pro dlouhodobé obsluhování velkého množství hráčů.

U poskytovatelů cloudových serverů je možné vybrat množství systémových prostředků, které bude mít aplikace k dispozici.
Pokud sledujeme vytížení herních serverů v čase, můžeme spatřit jisté vzory, například nárůst hráčů ve večerních hodinách.
Jedná-li se o velký rozdíl v množství uživatelů, je nutné dynamicky navyšovat systémové prostředky \cite{efficient_resources}.
Stejně jako dříve zmíněný problém se i tento týká převážně aplikací pro velké množství uživatelů, v rámci této práce tedy tento problém není uvažován.
Již přidělené prostředky nemůže vytvořený systém nijak ovlivnit, jejich výběr bude tedy ponechán na uživateli před spuštěním.

Důležitým prvkem kteréhokoliv systému je zabezpečení. Aplikace musí být s důrazem na bezpečnost nejen provozována,
ale i vytvářena \cite{newcombe_2012}. Bezpečnostní nedostatek může pro potenciálního útočníka znamenat možnost neoprávněného vstupu do systému.
Budou tedy prozkoumána dostupná bezpečností řešení pro cloudové aplikace.

Obraz systému musí být schopný nainstalovat a spustit herní server automaticky, případně pouze s nezbytně nutnou interakcí uživatele.
Bude proveden průzkum dostupných možností pro automatickou instalaci herních serverů za účelem výběru vhodného řešení.
Spouštění i zastavování herních serverů musí být plně automatizovatelné.

\section{Operační systém}

\section{Správce herních serverů}