\chapter{Publikace výsledků}

V následujících sekcích jsou popsány metody zveřejnění vytvořeného obrazu systému včetně zdrojových kódů
a příslušné dokumentace.

\section{TurnKey Hub}

Webová aplikace TurnKey Hub \cite{turnkey_hub} slouží k jednoduchému nasazování obrazů systémů založených na TurnKey GNU/Linux ve službě Amazon EC2.
Obsahuje obrazy schválené správci TurnKey GNU/Linux a umožňuje uživatelům jednoduše vytvářet instance systémů v cloudovém prostředí.

Vytvořený obraz systému pro správu herních serverů prošel první fází schvalovacího procesu a správci nemají výhrady k jeho účelu, s vysokou pravděpodobností
jej tedy bude možné zařadit mezi schválené systémy.
V době psaní bakalářské práce probíhá migrace TurnKey GNU/Linux obrazů na novou verzi systému, přijímání nových obrazů je tedy dočasně pozastaveno.

\section{GitHub}

Zdrojové kódy programu pro správu herních serverů \cite{github_linux_gameservers} a obrazu TurnKey \cite{github_turnkey_gameserver}
jsou k dispozici na serveru GitHub. Toto umístění umožňuje jejich další rozvoj v případě zájmu komunity a poskytuje i
možnost komerčního využití.

Dokumentace k jednotlivým částem je začlěnena do repozitářů ve formě \mintinline{shell}{README} souboru, který je automaticky zobrazen
pod seznamem souborů v repozitáři. Obsahuje také popis využití jednotlivých komponent včetně praktických ukázek pro jejich zprovoznění.