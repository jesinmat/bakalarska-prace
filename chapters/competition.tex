\chapter[Současná řešení]{Současná řešení}

V této části budou prozkoumána existující řešení pro vytváření herních serverů. Tyto aplikace zpravidla podporují instalaci na různé cloudové operační systémy,
které nejsou pro následný provoz herního serveru významné a nebudou tedy v této kapitole rozlišovány. Zhodnocení operačních systémů bude proto provedeno v následující kapitole.

Vzhledem k velkému množství herních serverů existuje také mnoho způsobu jejich provozu a údržby. Aplikace nabízené přes platformu Steam
lze například spravovat pomocí nástroje SteamCMD \cite{steamcmd}, avšak tento způsob neumožňuje provozovat herní servery jiných společností.
Z tohoto důvodu existují aplikace na správu herních serverů, které umožňují spouštět servery od mnoha vydavatelů uživatelsky přívětivou cestou.
Tyto aplikace jsou zpravidla provozovány v samostatném operačním systému, aby byl zajištěn plynulý chod.
Mnohé z nich podporují systém plateb za provoz herních serverů a jsou tak vhodné pro komerční využití.

\section{GameCP}

Jedním z pokročilých nástrojů pro správu herních serverů je Game Control Panel \cite{gamecp}. Jedná se o administrátorský nástroj pro společnosti
zabývající se pronájmem herních serverů. Tato aplikace umožňuje uživatelům spravovat velké množství herních serverů pomocí webového klienta,
který je přehledný a snadno přizpůsobitelný. Aplikace podporuje možnost evidence plateb za herní servery a poskytuje prostředí pro poskytování
podpory klientům, kteří si herní servery pronajímají. Kromě herních serverů podporuje nástroj také práci s programy pro komunikaci, jako jsou TeamSpeak
nebo Ventrilo. Jedná se o placenou aplikaci s měsíčním předplatným bez dostupného zdrojového kódu.

\section{Pterodactyl}

Pterodactyl \cite{pterodactyl} je dalším z nástrojů, který poskytuje přehledné webové rozhraní pro vytváření herních serverů.
Podporuje přibližně 40 herních serverů a umožňuje uživatelům vytvářet instance serverů pomocí Docker kontejnerů. Jedná se o bezplatný nástroj
s volně dostupným zdrojovým kódem, který je zaměřený na provoz libovolného množství herních serverů na jednom zařízení. Tento systém je aktivně vyvíjen komunitou
a zakládá si na jednoduchosti a kvalitní dokumentaci. API umožňuje komplexní přístup k jednotlivým funkcím a nabízí tak možnost automatizace
vytváření serverů.

\section{GameserverApp}

Dalším z existujících placených nástrojů je GameserverApp \cite{gameserverapp}. Jedná se o systém zaměřený na propojení komunikační platformy Discord s herními servery.
Umožňuje zpoplatnění herních funkcí a poskytování speciálních funkcí na plaftormě Discord platícím hráčům. Podporuje automatizované vytváření herních serverů, nabízí však jen
minimální množství her. Pro uživatele bez vlastního serveru poskytuje také pronájem vlastních serverů. Nástroj je zpoplatněný s neveřejným zdrojovým kódem.
