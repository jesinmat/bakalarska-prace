\chapter[Současná řešení]{Současná řešení}

V této části budou prozkoumána existující řešení pro vytváření herních serverů.
Vzhledem k velkému množství herních serverů existuje také mnoho způsobu jejich provozu a údržby. Aplikace nabízené přes platformu Steam
lze například spravovat pomocí nástroje SteamCMD \cite{steamcmd}, avšak tento způsob neumožňuje provozovat herní servery jiných společností.
Z tohoto důvodu existují aplikace na správu herních serverů, které umožňují spouštět servery od mnoha společností uživatelsky přívětivou cestou.
Tyto aplikace jsou zpravidla provozovány v samostatném operačním systému, aby byl zajištěn plynulý chod.
Mezi existující nástroje se řadí například:
\begin{itemize}
    \item GameCP -- placený nástroj pro správu herních serverů,
    \item Pterodactyl -- aplikace pro správu mnoha herních serverů na různých systémech,
    \item PufferPanel -- podobná aplikace jako Pterodactyl, s menším množstvím podporovaných her,
    \item LinuxGSM -- sada skriptů pro správu herních serverů,
    \item Open Game Panel -- správce herních serverů, nyní již neudržovaný.
\end{itemize}
Mnohé z těchto aplikací jsou vhodné pro správu velkého množství herních serverů, které běží na různých systémech. Díky webovému rozhraní umožňují snadný
přístup pro administrátory, případně i zavedení platebního systému pro zákazníky. Jsou tedy vhodné převážně pro komerční subjekty, které se zabývají
provozem herních serverů pro zákazníky. Většina jejich funkcí je pro tuto práci zbytečná.