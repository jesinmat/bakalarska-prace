\chapter{Testování}

Tato kapitola se zaměřuje na testování součástí vytvořeného systému. Testy byly prováděny
manuálně, aby byla zajištěna kontrola uživatelského rozhraní. Cílem těchto testů je určit,
zda je systém funkční a splňuje stanovené požadavky.
Testování v cloudovém prostředí probíhalo na platformě OpenStack, která podporuje spuštění
inicializačních skriptů před prvním spuštěním systému.
V následujících sekcích je uvedeno několik
vybraných testů.

\section{Automatické sestavení systému}

Tento test ověřuje, zda je možné automaticky sestavit obraz operačního systému. Pomocí vývojového prostředí
TKLDev dochází k vytvoření obrazu ve formátu ISO, který je dále upravován pro použití u poskytovatelů
cloudových služeb. Sestavení nového obrazu operačního systému je dále nutné provádět pouze z důvodu
aktualizace na novou verzi systému Debian. Vzhledem k tomu, že k vydání nové verze dochází přibližně
v intervalu dvou let \cite{debian_release_stats}, nebylo vhodné tento test automatizovat.

\section{Interaktivní instalace a konfigurace}

Tento test probíhá v prostředí s interaktivním přístupem. Ověřuje se, zda je systém možné interaktivně nainstalovat, provést jeho
prvotní konfiguraci, vybrat požadovaný herní server a zadat údaje potřebné k jeho spuštění. Po skončení tohoto testu
je herní server spuštěný a lze se k němu připojit. Vhodným prostředím pro tento test je lokální virtuální stroj.
Testuje se obraz ve formátu ISO.

\section{Interaktivní konfigurace}

Test ověřuje správnou funkčnost interaktivní konfigurace poté, co dojde k automatické instalaci bez zásahu uživatele.
Tento případ nastává, pokud uživatel spustí obraz v cloudovém prostředí bez poskytnutí inicializačních údajů. Systém
musí být schopen provést prvotní spuštění a konfiguraci nezbytných součástí bez zásahu uživatele.
Po prvním přihlášení vyplní uživatel nezbytné údaje a následně dojde a k instalaci a spuštění herního serveru.
% Během tohoto spuštění dojde k vygenerování náhodného hesla pro administrátorský účet. Uživatel se poté zpravidla přihlašuje
% pomocí SSH klienta s přednastaveným klíčem, který byl zvolen před prvním spuštěním v cloudové službě.
% Při prvním přihlášení dojde ke spuštění 

\section{Automatická konfigurace}

V průběhu tohoto testu dochází k ověření správné funkce systému při plně automatickém provozu bez zásahu uživatele.
Veškeré potřebné údaje jsou před prvním spuštěním systému předány pomocí inicializačního skriptu. Dojde k automatickému nastavení
všech potřebných parametrů a následně ke spuštění vybraného herního serveru. 
