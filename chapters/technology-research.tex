\chapter{Rešerše technologií}

V této kapitole je popsán výběr technologií vhodných pro tvorbu požadovaného systému.
Vzhledem k množství existujících her a četnosti jejich aktualizací není vhodné vytvářet vlastní
systém pro jejich správu, bude tedy nezbytné využít již existujících správců herních serverů.

\section{Operační systém}

Tato sekce uvede možné operační systémy pro provoz herních serverů v cloudovém prostředí a zhodnotí
jejich výhody a nevýhody. Budou zde zhodnoceny pouze obecně rozšířené serverové operační systémy \cite{server_os_share} a jejich deriváty,
aby byla zajištěna podpora a maximální možná bezpečnost.

\subsection{Windows Server}

Jeden z nejrozšířenějších operačních systémů pro serverové využití. Jedná se o komplexní systém s relativně velkými požadavky na
systémové prostředky, který je vhodný i pro rozsáhlé aplikace. Pravidelné aktualizace zajišťují bezpečnost, k dispozici je technická
podpora pro zákazníky. Windows Server je systém s uzavřeným zdrojovým kódem a pro jeho provozování je nutné zakoupit licenci,
není tedy vhodný pro spuštění jednoduchého herního serveru.

\subsection{Linux}



\section{Správce herních serverů}
Nástroj LinuxGSM je zaměřen na jednoduchou instalaci a správu herních serverů pomocí skriptů z prostředí příkazové řádky \cite{linuxgsm}. 