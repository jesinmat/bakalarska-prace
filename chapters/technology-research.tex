\chapter{Rešerše technologií}

V této kapitole je popsán výběr technologií vhodných pro tvorbu požadovaného systému.
Vzhledem k množství existujících her a četnosti jejich aktualizací není vhodné vytvářet vlastní
systém pro jejich správu, bude tedy nezbytné využít již existujících správců herních serverů a nalézt vhodný
operační systém.

\section{Operační systém}

Tato sekce uvede možné operační systémy pro provoz herních serverů v cloudovém prostředí a zhodnotí
jejich výhody a nevýhody. Budou zde zhodnoceny pouze obecně rozšířené serverové operační systémy \cite{server_os_share} a jejich deriváty,
aby byla zajištěna podpora a maximální možná bezpečnost. Přestože je možné provozovat herní server i na operačním systému
určeném pro osobní počítače, tyto systémy často obsahují grafické rozhraní a nabízí pouze interaktivní instalaci, která
pro hledané využití není vhodná.


\subsection{Windows Server}

Jeden z nejrozšířenějších operačních systémů pro serverové využití, který je vhodný i pro rozsáhlé aplikace.
V porovnání s ostatními systémy se Windows Server vyznačuje vysokými požadavky na
systémové prostředky \cite{windows_server_reqs}. Pravidelné aktualizace zajišťují bezpečnost, k dispozici je technická
podpora pro zákazníky. Windows Server je systém s uzavřeným zdrojovým kódem a pro jeho provozování je nutné zakoupit licenci,
není tedy vhodný pro spuštění jednoduchého herního serveru.

\subsection{Linux}

Vzhledem k množství existujících distribucí Linuxu bude v této části uvedeno pouze několik vzorových distribucí.
Během rešerše vhodných operačních systémů bylo zkoumáno množství distribucí, které jsou použitelné pro provoz
herních serverů. Mnohé z nich jsou založeny na podobných principech, pro účely této kapitoly tedy nemá smysl uvádět všechny.

\subsubsection{Ubuntu Server}

Tento pokročilý operační systém pro provoz serverů, vyvíjený společností Canonical, se vyznačuje širokou podporou nástrojů a pravidelnými bezpečnostními
aktualizacemi. Podobně jako ostatní distribuce Linuxu i tato nabízí otevřený zdrojový kód a je volně využitelná. I přes uvedené výhody se však stále jedná o
rozsáhlý systém, vyžadující značné množství systémových prostředků \cite{ubuntu_server_reqs}.

\subsubsection{Red Hat Enterprise Linux}

% Red Hat Enterprise Linux nabízí ... 
TODO

\subsubsection{TurnKey GNU/Linux}

Operační systém TurnKey GNU/Linux vychází z rozšířené distribuce Debian. Jedná se o distribuci Linuxu zaměřenou na nasazení jedné aplikace v cloudovém prostředí
či na virtuálním stroji, která nabízí možnost automatické instalace i počáteční konfigurace. Úzké zaměření umožňuje snížit nároky na systémové prostředky
při zachování pouze nezbytných součástí pro bezpečný provoz v cloudovém prostředí. Díky otevřenému zdrojovému kódu a svobodné licenci je možné
tento systém provozovat bez omezení i pro komerční využití. Pro vývojáře je k dispozici dokumentace, pomocí které lze do systému snadno zakomponovat
nové aplikace.

\section{Sestavení obrazu systému}

K sestavení obrazu systému včetně součásti k provozu herního serveru je nutné zvolit vhodný nástroj. Takový nástroj
musí být schopný automaticky vytvořit spustitelný obraz systému, a to nejen pro provoz v cloudovém prostředí, ale i
v interaktivním prostředí virtuálního stroje. K tomuto účelu lze použít například nástroj Packer \cite{packer},
který nabízí automatizovatelné sestavení potřebných obrazů a umožňuje
přidat podporu pro nové formáty pomocí zásuvných modulů.

TurnKey GNU/Linux však již nabízí nástroje pro sestavení obrazů s vlastní aplikací. Poskytované skripty podporují tvorbu formátů
pro virtuální stroje i pro množství poskytovatelů cloudových služeb a poskytují možnost plné automatizace procesu sestavení.
Pro požadované využití jsou tedy vhodným prostředkem. Vzhledem ke zmíněným vlastnostem bude tento systém využit pro vytvoření
požadovaného obrazu.

\section{Správce herních serverů}

Nezbytnou součástí pro správu velkého množství her je aplikace pro správu herních serverů. V předchozích kapitolách
byla uvedena některá existující řešení tohoto problému, která se vyznačovala velkým množstvím funkcí včetně
evidence plateb, propojení herních serverů se sociálními platformami či pokročilým webovým rozhraním.
Tyto funkce však nejsou pro provoz herního serveru nezbytně nutné a proto nebudou podobné komplexní systémy v této kapitole zahrnuty.

\subsection{LinuxGSM}

Nástroj LinuxGSM \cite{linuxgsm} je zaměřen na jednoduchou instalaci a správu herních serverů pomocí skriptů z prostředí příkazové řádky.
Jedná se o minimalistický systém skriptů, který umožňuje v rámci aktuálního systému stahovat, instalovat a provozovat herní servery.
Tento bezplatný nástroj podporuje více než 100 různých herních serverů a díky otevřenému zdrojovému kódu je uživatelům umožněno aktualizovat
a přidávat nové herní tituly. Mezi pokročilé funkce patří zálohování serverů, monitorování jejich stavu či zasílání upozornění při definované akci.