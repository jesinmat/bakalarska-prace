\begin{introduction}

Počítačové hry se těší velké popularitě. S rozšířením internetu začaly vznikat také hry
pro více hráčů, které se rychle dostaly do čela žebříčků oblíbenosti a dnes jsou zábavou pro stovky milionů
hráčů po celém světě.

Mnohé z těchto her umožňují uživatelům vytvořit vlastní herní servery, na kterých je možné hrát s přáteli
či jinou komunitou.
Pokud chce uživatel zprovoznit herní server, měl by takový postup být jednoduchý a rychlý.
Je možné provozovat server na vlastním počítači, zde je však kvalita herního zážitku 
ovlivněna konfigurací systému a internetovým připojením. Také je často potřeba pokročilého nastavení
směrovače, který herní server z domácí sítě zpřístupní do internetu.

Tato práce se zaměřuje na další možnost provozu těchto serverů, a to v cloudovém prostředí. Uživatel se tak nemusí zabývat 
kvalitou internetového připojení či manuálním nastavováním síťových prvků. Herní servery
pro menší počet hráčů jsou často vytvářeny a rušeny, nasazení v cloudu tedy představuje ideální způsob provozu,
kde jsou tyto operace jednoduché a automatizovatelné.

Cílem práce je vytvořit obraz systému, který bude pro toto použití vhodný. Uživatel bude mít možnost vybrat
požadovaný herní server a systém provede všechny operace potřebné k jeho zprovoznění.

Výsledek práce bude prospěšný zejména pro stávající uživatele cloudových služeb, kteří mají alespoň minimální
zkušenosti s nasazováním obrazů systémů. S minimální interakcí budou mít možnost spustit herní server dle svého
výběru bez složité instalace a konfigurace. Pokročilí uživatelé využijí možnosti automatizace celého procesu,
která jim zaručí rychlé spuštění vybraného serveru za pomoci několika příkazů.

Vytvořený obraz musí být snadno dostupný a zdokumentovaný, jednoduchý na nasazení, s minimální náročností 
na systémové prostředky. Bude jej také možné využít v komerčním prostředí.

\end{introduction}